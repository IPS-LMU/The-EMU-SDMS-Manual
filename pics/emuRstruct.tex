%!TEX root = ../diss.tex

\tikzstyle{robj}=[rectangle, thick, fill=robjblue,stroke=grey, text=white,
                  inner sep=0.1cm, rounded corners]

\tikzstyle{rfun}=[rectangle, thick, fill=rfungrey, text=black,
                  inner sep=0.1cm, rounded corners]


\begin{tikzpicture}[scale=0.8]
  
  
  \definecolor{myc1}{RGB}{27,158,119}
  \definecolor{myc2}{RGB}{217,95,2}

  %%%%%%%%%%%%%%%%%%%%%%%%%%%%%%
  %%%%%%%%%%%%%%%%%%%%%%%%%%%%%%
  % hierarchy

  %%%%%%%%%%%%%%%%%%%%%%%%%%%%%%
  % nodes

  % R nodes

  % boxes
  \draw[fill=gray, rounded corners] (-8,2.5) rectangle (8, -2.5);
  \node (emuR) at (-4.3, 2.1) {\texttt{emuR}: \texttt{emuDB} interaction funcs};

  \draw[rounded corners, dashed] (-8.5,3) rectangle (8.5, -3);
  \node (R) at (-8.5,3) {\pgftext{\includegraphics[scale=0.03]{pics/RlogoNew}}};

  % function nodes
  \node (loademuDB) at (0, -2) [rfun] {\texttt{load\_emuDB()}};

  \node (emuDBhandle) at (0, 0) [robj] {\texttt{emuDBhandle}};

  \node (gettrackdata) at (-5, -1) [rfun] {\texttt{get\_trackdata()}};

  \node (addlinkdef) at (-5, 1) [rfun] {\texttt{add\_linkDefinition()}};

  \node (dots) at (0, 2) [rfun] {\texttt{\dots}};

  \node (requeryhier) at (5, 1) [rfun] {\texttt{requery\_hier()}};

  \node (query) at (5, -1) [rfun] {\texttt{query()}};

  % \node[text width=5cm] (funs) at (4,0) {
  %     \begin{tcolorbox}[%
  %         title={},
  %         width=\textwidth
  %     ]
  %     \begin{itemize}
  %     \itemsep-10pt
  %       \item ...
  %       \item \texttt{dftSpectrum()}
  %       \item \texttt{forest()}
  %       \item \texttt{ksvF0()}
  %       \item ...
  %     \end{itemize}

  %     \end{tcolorbox}
  % };


  % files / db nodes
  \node[text width=7cm] (emuDB) at (-4.5,-8) {
      \begin{tcolorbox}[%
          title={exampleDB\_emuDB},
          width=\textwidth
      ]
      \begin{tikzpicture}[scale=0.8]
        \altentry{\textit{exampleDB\_DBconfig.json}}{1}
        \altentry{\textbf{0002\_ses/}}{1}
        \altentry{\textbf{bundle2\_bndl/}}{2}
        \altentry{\textit{bundle2.wav}}{3}
        \altentry{\textit{\dots}}{3}
      \end{tikzpicture}

      \end{tcolorbox}
  };

  \node[text width=6cm] (relModel) at (4.7,-8) {
      \begin{tcolorbox}[%
          title={relational annot. struct.},
          width=\textwidth
      ] 
      \begin{small}
        % items
        \begin{tabular}{|c|c|c|}
        \hline
        \multicolumn{3}{|c|}{\textbf{items table}} \\
        \hline
        \dots & item\_id & \dots \\ 
        \hline
        \end{tabular}
        % labels
        \begin{tabular}{|c|c|c|c|}
        \hline
        \multicolumn{4}{|c|}{\textbf{labels table}} \\
        \hline
        \dots & item\_id & label & \dots \\ 
        \hline
        \end{tabular}
        % links
        \begin{tabular}{|c|c|c|c|}
        \hline
        \multicolumn{4}{|c|}{\textbf{links table}} \\
        \hline
        \dots & from\_id & to\_id  & \dots\\ 
        \hline
        \end{tabular}
        \\
        \\
        \dots
        \end{small}
      \end{tcolorbox}
  };





  %%%%%%%%%%%%%%%%%%%%%%%%%%%%%%
  % links
  \draw [<->, dotted, line width=1] (emuDB.east) to node[rotate=90, xshift=-1.3cm]{synchronize} (relModel.west); 

  \draw [->, line width=1] (emuDB) to (loademuDB);
  
  \draw [->, line width=1] (loademuDB) to (emuDBhandle);

  \draw [->, myc2, line width=1] (emuDBhandle) to [bend right=20] (emuDB);

  \draw [->, myc1, line width=1] (emuDBhandle) to [bend left=20] (relModel);

  % pointing to emuDBhandle
  \draw [->, myc2, line width=1] (gettrackdata) to  (emuDBhandle);

  \draw [->, myc2, line width=1] (addlinkdef) to  (emuDBhandle);

  \draw [->, myc2, line width=1] (dots) to  [bend right=20](emuDBhandle);

  \draw [->, myc1, line width=1] (dots) to  [bend left=20](emuDBhandle);

  \draw [->, myc1, line width=1] (requeryhier) to  (emuDBhandle);

  \draw [->, myc1, line width=1] (query) to  (emuDBhandle);

  % \draw [<->] (emuDB) to (cache);
 

  % \draw [dotted,<->] (emuDB.east) to [bend right=45] node[text width=2.3cm]{\tiny{\texttt{INPUT:} File Path(s)} \tiny{\texttt{OUTPUT:} Data File(s)}} (funs.south);

  
  
\end{tikzpicture}
